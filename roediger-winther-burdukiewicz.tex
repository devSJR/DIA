% !TeX root = RJwrapper.tex
\title{Image Analysis Software for R: a Review}
\author{Stefan R\"{o}diger, Hinrich Winther and Micha\l{} Burdukiewicz}

\maketitle

%An abstract of less than 150 words.
\abstract{
Management, display, and processing of biological and medical imaging data is 
an important task in life sciences and medical research. R is a powerful 
cross-platform, which can handle most task statistical computing tasks in 
the same environment. The aim of this mini-review is to give a brief overview 
about image processing software for the R statistical computing environment. 
When it comes to image analysis, R may appear to provide only few tools on the 
first sight. However, a systematic analysis of the existing packages shows 
shows that a huge potential for numerous applications.
}

\section{Introduction}
There are numerous software tools that have been made available for digital 
image processing. The user groups are deliberately broad including biologists, 
physicians and related life scientists. Often the knowledge of image processing 
is gained by self-study. In particular, learning of several programming 
languages may hamper the scientist to focus on their scientific aim. R \citep{R} is \textit{de facto} 
the \textit{lingua franca} of statistical bioinformatics and 
therefore used in numerous research disciplines \citep{rodiger_r_2015}. It is a 
powerful tool for statistical data analysis. It comes to no 
surprise that software packages for digital image processing have been 
implemented. In this review, we give an overview of the R ecosystem about 
which software packages exist and which deficits they may expose in 
comparison to other software packages. We aimed to aggregate information about 
R packages available on CRAN, BioConductor or github.

We performed two image processing case studies were we applied selected 
packages for immunoflourescence image analysis and RMI data. 

There are numerous software packages for the analysis of image data 
\citep{wiesmann_review_2015}. However, R is quite functional when it comes to 
digital image analysis.

image processing capabilities of Cell-ID and data analysis by the statistical 
programming framework R for quantifying various cellular features (e.g., volume, 
total and subcellular fluorescence localization) from sets of microscope 
images of individual cells \citep{bush_using_2012}


\section{Give me a title}

General image processing and analysis


\citep{tabelow_modeling_2012, tabelow_dti:_2014}

Murrel \citep{murrell_raster_2011}
\citep{mmand} \citep{clayden_mmand:_2016}

CRAN provides well established packages. These are \CRANpkg{jpeg} 
\citep{urbanek_jpeg:_2014} and \CRANpkg{png} \citep{urbanek_png:_2013} to 
read, write and display bitmap JPEG and PNG images, respectively. The 
development of the \CRANpkg{ripa} \citep{perciano_ripa:_2014} package was started in 2005 by Talita Perciano. 
This package can be used to processes and analyses RGB, LAN (multispectral) 
and AVIRIS (hyperspectral) images. Recent advances of \CRANpkg{ripa} make it a promising 
tool for analysis of large datasets. The vast amount of image data is becoming 
more and more and essential part of Big Data analysis pipelines. R is among 
the frequently used for data mining and analysis. It comes to no surprise the 
commercial and non-commercial entities make heavy use of R \citep{chen_big_2014}. \BIOpkg{EBImage} \citep{pau_ebimager_2010} 
is presumably the most comprehensive package and the foundation for many other 
R packages in the context of microscopy-based cellular assays \citep{gowen_near_2015}. This 
package offers tools to transform (e.g, rotate) the images, segment object 
(e.g., cells) and extract quantitative descriptors. The early version of 
\BIOpkg{EBImage} used the Magick++ interface to the ImageMagick image processing 
library \citep{sklyar_image_2006}.

\section{General image processing and analysis}

This section may contain a figure such as Figure~\ref{figure:bead}.

\begin{figure}[htbp]
  \centering
  \includegraphics[clip=true,trim=0.1cm 0.3cm 0.2cm 0.1cm, width=12cm]{bead}
  \caption{The logo of R.}
  \label{figure:bead}
\end{figure}

\section{segmentation}

\citep{holmes_interactive_2009}

\CRANpkg{adimpro} is a package for manipulation of digital images and the 
Propagation Separation approach for smoothing digital images \citep{polzehl_adaptive_2007}.
For example, image analysis is used for the detection and quantification of 
cell patterns and array technologies like microarrays and bead-based assays 
\citep{rodiger_highly_2013, willitzki_new_2012, willitzki_fully_2013, 
dunning_beadarray:_2006}.
Several software packages have been developed. imageJ belongs to the most 
used and cited tools. When it comes to R numerous packages exist, which can 
be readily integrated in the analysis routines \citep{frery_introduction_2013}.

The accuracy of image segmentation is a critical step in a computer-aided 
diagnosis systems. The recognition of mitotic cells and the classification of 
fluorescent patterns is heavily dependent on this. Immunofluorescent images 
of cell, such as Hep-2, exhibit a high variability due a wide range of staining 
patterns and intensity levels (FIGURES OF CELLS), the presence of mitotic 
cells and  artifacts. The later may be caused by uneven illumination and 
photo-bleaching effects \citep{tonti_automated_2015}.

Intensity inhomogeneity (bias field) is a common artefact in 
magnetic resonance (MR) images, which hinders successful automatic segmentation. \citep{ivanovska_efficient_2016}


\begin{example}
  x <- 1:10
  result <- myFunction(x)
\end{example}


\section{Applications}

\CRANpkg{CRImage} package \citep{failmezger_crimage:_2012} for tumor image analysis

\CRANpkg{AnalyzeFMRI} \citep{marchini_analyzefmri:_2002} and \CRANpkg{fmri} 
\citep{polzehl_fmri:_2007} and are packages for the analysis of Magnetic 
Resonance Imaging (MRI) and functional Magnetic Resonance Imaging (fMRI) data, 
respectively.

Eventually these de 

Others include \CRANpkg{dcemri} \citep{dunning_beadarray:_2006, frery_introduction_2013}.

\section{GUIs}

There exist also R graphical user interfaces \citep{rodiger_rkward:_2012} which can be used for for image processing. 
Bio7 is an integrated development environment based on the Eclipse Rich Client 
Platform (RCP). The main purpose of this tool is the modeling and analysis of 
ecological systems. However, Bio7 is not restricted to this discipline. T
he application contains GUIs and plugins for simulation and analysis tasks. 
Interestingly, one of these plugins is an adaption image application ImageJ 
and another is available for a bidirectional Java connection to R. This 
means that data can be transferred from and to ImageJ and R.

\section{Performance}

Micha\l{}, would you like to take this section?


Requirements for recent research include the rapid processing of massive amounts 
of image data (Mega to Tera byte scale) that modern technologies (e.g., 
microscopes, MRI scanner) produce nowadays. Preferably, affordable personal desktop
computers should be usable. R has several disadvantages when it comes to memory management
and GPU and CPU usage ...

\section{Summary}

Many scientist use and master R. We would like to raise awareness for the fact that
R provides sophisticated packages for digital image analysis. Added values for the
user are that there is less need to learn a new programming languages and that all
analysis can be performed in a consistent and cross-platform environment.


Table~\ref{table:packages} gives an overview of R packages currently available.

\begin{table}
\begin{center}
\begin{tabular}[c]{llll}
Package & Main function & Comment & Source\\
\BIOpkg{EBImage} & fancy stuff & well maintained & BioConductor\\
\CRANpkg{AnalyzeFMRI} & × & × &
\end{tabular}
\end{center}
\caption{\label{table:packages}
R packages.
}
\end{table}

\bibliography{roediger-winther-burdukiewicz}

\address{Stefan R\"odiger (corresponding author)\\
  Faculty of Natural Sciences\\
  Brandenburg University of Technology Cottbus--Senftenberg\\
  Senftenberg\\
  Germany}
\email{Stefan.Roediger@b-tu.de}

\address{Hinrich Winther\\
  Affiliation\\
  Address\\
  Country\\}
\email{author@work}

\address{Micha\l{} Burdukiewicz\\
  University of Wroclaw\\
  Faculty of Biotechnoloy\\
  Department of Genomics\\
  Wroclaw\\
  Poland}
\email{michalburdukiewicz@gmail.com}
